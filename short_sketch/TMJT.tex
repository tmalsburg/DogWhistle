\documentclass[11pt]{article}
\usepackage{helvet}
\renewcommand{\familydefault}{\sfdefault}
\usepackage{color}
\usepackage[utf8]{inputenc}
\usepackage[english]{babel}
\usepackage{enumerate}
\usepackage[backend=biber, style=authoryear, maxnames=10, url=false, doi=false, eprint=false, natbib]{biblatex}
\addbibresource{../bibliography/bibliography.bib}

\sloppy

% SECTION FOR SWITCHING ON AND OFF THE HINTS FOR FILLING IN
% INFORMATION show hints by dfg for filling in the template:
\newcommand{\dfgcomments}[1]{\textcolor{green}{#1}}
% un-comment to switch off hints
\renewcommand{\dfgcomments}[1]{}

% Start section numbering at 3, as in official template needs to be
% defined here and in main document so we can compile both stand-alone
% and the full proposal:
\setcounter{section}{3}
\setcounter{subsection}{0}

% Read configuration code that should apply to all projects:
% adapt margins so they correspond to DFG template 53_120
\usepackage[lmargin=2.5cm, rmargin=2.5cm, tmargin=3cm, bmargin=2cm]{geometry}
% babel wants csquotes
\usepackage{csquotes}
\usepackage{todonotes}

\definecolor{jk}{rgb}{0.60,0.75,0.95}
\newcommand{\jonastodo}[2][jk]{\todo[color=#1,size=\scriptsize]{\textbf{#1:} #2}}

\definecolor{jt}{rgb}{1.0, 0.75, 0.0}
\newcommand{\judithtodo}[2][jt]{\todo[color=#1,size=\scriptsize]{\textbf{#1:} #2}}

\usepackage{graphicx}
\usepackage{tipa}

\usepackage{ulem}
\normalem  % jk added this to get italics for \emph

% adapt section heading font size, so they corresond to DFG template 53_120
\usepackage{titlesec}
\titleformat{\section}{\normalsize\bfseries}{\thesection}{1em}{}
\titleformat{\subsection}{\normalsize\bfseries}{\thesubsection}{1em}{}

\newcommand{\sfbcomments}[1]{\textcolor{blue}{#1}}
\renewcommand{\sfbcomments}[1]{}

\newcommand{\owncomment}[1]{\textcolor{red}{#1}}
%\renewcommand{\owncomment}[1]{}

\newcommand{\areaConst}{C}
\newcommand{\areaAdapt}{A}
\newcommand{\areaMap}{M}
 

\begin{document}

% Checklist:
% - [ ] Incorporate feedback from reviewers. Especially important:
%       Reconsider using or explain discipline-specific terminology.
% - [ ] Mention the language varieties and corpora investigated.
% - [ ] Make sure we connect to the main topics of the CRC, ref
%   systems (check), but also their transformation.  What else?

%\section{Teilprojekte}
%\section{Projects}

\subsection{Code and title}

A5 -- Subjective beliefs and misaligned reference systems: A unifying theory of gender biases and dog whistles

% How social meaning arises from miscalibrated reference systems
% How social meaning arises from the tension of rational inference and misaligned reference systems
% Misaligned reference systems and social meaning: The case of gender biases and dog whistle expressions

\subsection{Project leaders}

\begin{tabular}{p{.5\textwidth}p{.5\textwidth}}
  Jun.-Prof.\ Dr.\ Titus von der Malsburg \par
  02.04.1977 \par
  Department of English Linguistics \par
  Keplerstr.\ 17, 70174 Stuttgart \par
  0711 / 685-8xxx \par
  titus.vondermalsburg@ling.uni-stuttgart.de
  &
  Prof.\ Dr.\ Judith Tonhauser \par
  19.11.1974 \par
  Department of English Linguistics \par
  Keplerstr.\ 17, 70174 Stuttgart \par
  0711 / 685-83121 \par
  judith.tonhauser@ling.uni-stuttgart.de
\end{tabular}

\subsection{Project description}

% Describe the main research question, the current state of the art,
% project-related preliminary work, the work programme and
% methodology, and integration in the Collaborative Research
% Centre. The information provided in this section should be able to
% stand on its own and be understandable, coherent and reviewable
% without the need to read additional documents.

% Section 3.3.1 (½ page)
\subsubsection{Summary and main research question}
% Should be able to stand alone. Include specific research question,
% connect to broader research question, area and theme.

The reference systems of participants in a discourse can rarely, if ever, be assumed to perfectly align, which creates room for ambiguity and misunderstandings.  This is particularly the case with social meaning, i.e. meaning that informs us about where interlocutors position themselves in the social coordinate system and which arises implicitly and fluidly from how we express ourselves.  The goal of this project is to use empirical, experimental, and computationally methods to investigate two topical instances of social meaning, linguistic gender biases and dog whistle expressions, and to explain how they naturally arise from the assumption of miscalibrated reference systems when viewed through the lens of rational pragmatic inference.

Gender bias is exemplified by a speaker referring to the future president as “he” even though the president is expected to be female, or by a listener experiencing processing difficult upon hearing “she” in reference to a likely female future president.  So-called dog whistles are expressions that encode a literal, typically benign meaning, while also carrying a hidden, more controversial, or even offensive non-literal social meaning that only a subset of the audience is receptive to.  An example is the use of “inner city crime” in the United States which is argued to carry racial overtones by implying crime committed predominantly by black people.  These two phenomena are superficially different in that gender biases are assumed to be unintended and unconscious whereas dog whistles are believed to reflect a conscious strategy that exploits blind spots in the audience’s reference system.  This analysis, however, may not hold up to scrutiny.  Gendered language can be intentionally used to signal potentially controversial social views (e.g., German “Fräulein”, roughly \textit{missy}) and dog whistles can be unintentional expressions of unconscious biases (e.g., German “Clan-Kriminalität” when referring to organized crime conducted by extended Arabic families).

This project contends that both phenomena, gender biases and dog whistles, can be understood as products of the same underlying dynamics, namely the interaction of rational pragmatic inference with miscalibrated reference systems.  The overarching goal of this project is then to investigate whether an existing formal account of pragmatic inference, namely the rational speech act model \parencite[RSA, ][]{FrankGoodman2012} can explain gender biases and dog whistles as naturally arising from the framework’s independently motivated assumptions about efficient communication.  Gender biases and especially dog whistles have been predominantly studied in English and in the context of US culture.  A secondary goal of this project is therefore to empirically investigate both phenomena in German.

% Section 3.3.2 (¾ page)
\subsubsection{Current state of research and preliminary work}
% This is a longer text citing both external work and own work .
% References to research outside the CRC will be included in general
% section.

% TODO:
% - [ ] Clarify that plain RSA can handle neither dog whistles nor
%   gender biases, a clear shortcoming of the model that we will address.

% Something on gender biases:

A contentious issue in German is the generic masculine, e.g., the use of the masculine noun “die Lehrer” when talking about a set of male and female teachers.  Many studies have demonstrated that the generic masculine is interpreted with a bias toward male referents \parencite[e.g.,][]{GygaxEtAl2008, Kusterle2011}, an observation that has inspired the introduction of novel gender-neutral forms such as those containing the so-called gender star (“Lehrer*innen”).  However, there are two key issue in this area research: 1.\ There is currently no clearly articulated theory that explains how these biases arise and where in the system they are rooted.  2.\ The debate does not stand on solid empirical footing.  Experiments typically expose participants to many instances of the generic masculine along with gender-neutral forms and feminine forms.  It is practically certain that participants develop some awareness of what is being tested and this may alter their behavior.  For instance, a salient gender-neutral alternative may lead participants to interpret the generic masculine as more male-oriented than they usually would; or participants might interpret the generic more neutral due to social pressure.  Further, most studies tested local undergraduate populations and results may therefore not generalize.  With regard to gender biases, this project therefore identifies two desiderata: 1.\ A theoretical account that explains the use and interpretation of the generic masculine and its alternatives.  2.\ An improved empirical basis for understanding the generic masculine.

The starting point with regard to dog whistles is different: There is a rich theoretical literature but little empirical and no experimental data (none known to us).  One clearly articulated proposal situates dog whistles in the framework of Bayesian signaling games \parencite{Burnett2017} and contents that dog whistles are used to signal one’s persona and ideologies to a subset of the audience while keeping others in the dark \parencite{HendersonMcCready2018, HendersonMcCready2019}.  While theoretically attractive and precisely formulated, this proposal has not yet been thoroughly evaluated.  It is also not clear to us that it captures the full breadth of the phenomenon.  For instance, when George Bush says “family values”, one could argue that he is not plausibly denying his Christian identity which is widely known.  One alternative interpretation is that speakers’ choice of language rather displays their biases much like the choice of masculine expressions does in the case of gender biases especially when explicit (e.g., racist) language is off limits.  In this project we will formally implement this hypothesis as a variant of the RSA model and compare it to the \citeauthor{HendersonMcCready2019} proposal using corpus data and experimental data combined with Bayesian model comparison.

Both PIs contribute crucial experience to this project.  TODO Tonhauser is an expert ….  Von der Malsburg used large-scale online experimentation to study gender biases during the 2016 US presidential campaign and found that speakers avoided feminine pronouns to refer to the future president even when they believed that the female candidate (Clinton) was going to win (\cite{MalsburgEtAl2020}, see also \cite{PoppelsEtAl2021CUNY, PozniakBurnett2021}).  This study further found that the strength of these biases varies as a function of age, political alignment, and education, suggesting that data from just undergraduate populations may indeed be misleading.  \textcite{BoyceEtAl2019LSA} generalized these findings and demonstrated that feminine pronouns are systematically under-used and that comprehenders systematically under-infer female gender both relative to perceived real-world statistics.  Von der Malsburg is also an expert for Bayesian inference \parencite{MorganEtAl2020, MeziereEtAl2021} which puts us in an ideal the position to formally implement our theoretical ideas and to evaluate them using state-of-the-art Bayesian model comparison tools.

% https://linguisticpulse.com/2014/06/12/demystifying-dog-whistle-racism/
%
% Works to mention: Stanley book, Beaver/Stanley 2017 article, Nunber: the social life of slurs

\paragraph{References to work outside the CRC}

% This bibliography will list only publications that do not have
% TMJT-own or TMJT-own-other keywords in the .bib file (see example in
% template):

\printbibliography[heading=none,notkeyword={TMJT-own},notkeyword={TMJT-own-other}]

% Section 3.3.3 (¾ page)
\subsubsection{Work programme and role within the proposed Collaborative Research Centre}

% Long-term questions:
%
% 1. Which expressions index beliefs for which groups of
% language users? Which meanings do they convey / which
% connotations/ideologies do they index?
%
% 2. Which linguistic and extralinguistic properties taxonomize
% these expressions? Which meanings/connotations/ideologies are
% conventionally coded, which ones arise from interlocutor
% beliefs/biases/stereotypes?
%
% 3. What does a predictive analysis of these expressions look
% like? How can such an analysis be validated in large, unconstrained
% datasets?
%
% Questions to be addressed in phase 1:
%
% - German expressions around the topics of gender
%
% code words/dogwhistle: {\em Tante, Tussi}
%
% gendered nouns (generic masculine nouns, inclusive forms with `Binnen-I', *, :)
%
% - Methods: annotation of naturally occurring examples
% (Bundestagscorpus, newspapers), experiments to diagnose properties
% of the content, social perception, language comprehension, and
% language processing
%
% - Computational modelling
%
% - Project is also concerned with challenges to successful
% communication, signaling persona and ideology, signaling intended
% action and in/out-group building
%
% - written or spoken language, or both?

Ignore the text below.  Just a scratch pad for text blocks.

Specifically, we will embed this issue in the RSA model of rational pragmatic inference and evaluate it using corpus data and experimental comprehension and production data.  Within this framework we will also investigate the alternatives to the generic masculine since there is no strong a priori reason to believe that they are necessarily neutral (e.g., “Lehrer*innen” may still be male-biased, or female-biased).

- Explain
- Explain online methods.

This project will address points 1. and 2. through large-scale online experiments that tap into a much wider participant population than typical lab-based studies.  Online experiments also open the door to so-called one-shot designs in which each participant sees just one stimulis item and hence cannot develop an experiment-specific strategy.  Point 3. will be adressed by extending the rational speech act model to see which assumptions are necessary to explain the observed data.  Given the fully spelled-out mathematical nature of this model, we will be able to deploy the full arsenal of Bayesian model comparison tools for this purpose.

The goals of this project with respect to dog whistle are therefore the following: 1. We will address the thin empricial foundation for studying dog whistles in German by establishing a corpus of real-world examples from political speeches and public discourse.  2. We formulate a model explaining the use and interpretation of dog whistles.  The aim is to treat dog whistles within the same model as will be use for gender biases (see above) and to contrast it with other models, again using the tools of Bayesian model comparison.  3. Finally, we will evaluate the model predictions with regard to the production and interpretation of dog whistles through psycholinguistic experiments.

% Section 3.4 (¾ page)
\subsection{Project-related publications by participating researchers}
% - [ ] 10 per project, not per PI.
% - [ ] Must be related to the project (mention in 3.3.2, if suitable).
% - [ ] Separated into ‘published/accepted’ and other.

%{\bf List only publications by the participating researchers whose topics are directly related to the proposed individual project and which are publicly available. max 10 total}

\dfgcomments{\begin{enumerate}[a)]
  \item articles which at the time of draft proposal submission have been published or officially accepted  by  publication  outlets  with  scientific  quality  assurance,  and  book  publications;
  \item other publications;
  \item patents (subdivided into pending and issued).
  \end{enumerate}}

\dfgcomments{The total number of works listed under a) and b) combined may not exceed ten. When listing papers that have been officially accepted for publication but not yet published, the manuscript  and  the  publisher’s  dated  acknowledgement  of  acceptance  must  be  submitted  via  elan. The DFG will forward an electronic copy of the draft proposal -- and a print copy upon request -- to the consultation panel.}

\paragraph{a) Peer-reviewed published or accepted articles}

% This bibliography will list only publications that have TMJT-own as
% a keywords entry in the .bib file (see example in template)
% publications that have not been cited already above can be inserted
% by: \nocite{Scientist/Researcher:2019} redundant if they have been
% cited in the preliminary work session above.

% Judith:
\nocite{StevensEtAl2017}
\nocite{TonhauserEtAl2013}
\nocite{TonhauserEtAl2018}
\nocite{DeMarneffeEtAl2019}
\nocite{Tonhauser2012}
\nocite{DegenTonhauserMS}
\nocite{KiparskyTonhauser2013}

% Titus:
\nocite{MalsburgEtAl2020}
\nocite{MalsburgVasishth2013}
\nocite{PaapeEtAl2020}

\printbibliography[heading=none,keyword={TMJT-own}]

\paragraph{b) Other publications}
% This bibliography will list only publications that have
% TMJT-own-other as a keywords entry in the .bib file (see example in template)
% publications that have not been cited already above can be inserted by:
% \nocite{Mohler:2001}
% redundant if they have been cited in the preliminary work session above

\printbibliography[heading=none,keyword={TMJT-own-other}]

% Section 3.5 ‘Project funding’ (3 lines)
\subsection{Project funding} % 3 lines
% We will provide wording after strategic meeting with the Rektorat.

(To be filled in at a later date.)

% Two PhDs, one working on gender one on dig whistles.

\end{document}